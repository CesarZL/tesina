\section{Conclusiones y Trabajos Futuros}
Para finalizar con este documento es necesario abordar con las conclusiones obtenidas a lo largo de este proceso de desarrollo, también es importante tener en cuenta el trabajo a futuro y proponer mejoras en todo lo que no pudo realizarse, teniendo en cuenta siempre un progreso notable en el proyecto. 

\subsection{Conclusiones del Proyecto}
Este proyecto tuvo como objetivo principal desarrollar un sistema de formateo de artículos científicos en formato LaTeX, que permita al usuario subir plantillas LaTeX, crear y editar artículos, y compilarlos en archivos PDF. Esto con el objetivo de facilitar la creación y edición de artículos en la comunidad científica y académica, debido a que existen diferentes plantillas LaTeX para diferentes revistas y conferencias, lo que dificulta el proceso de formateo de los artículos, siendo poco eficiente tener que cambiar manualmente el formato de los artículos para cumplir con los requisitos de las plantillas.

Durante el desarrollo del proyecto, se llevaron a cabo diversas actividades de investigación, diseño, desarrollo, implementación y pruebas para garantizar el funcionamiento del sistema. Se establecieron procesos y procedimientos detallados para cada etapa del proyecto para minimizar los problemas y garantizar la calidad del sistema. 

Uno de los principales desafíos durante el desarrollo fue la integración de diversas tecnologías y que funcionaran de manera conjunta. La combinación tecnologías permitió crear una plataforma robusta y versátil para la creación y edición de documentos. Sin embargo, se encontraron dificultades en la configuración inicial del entorno de desarrollo y en la integración de algunas herramientas, lo que requirió un esfuerzo adicional para resolver.

A pesar de los desafíos, el proyecto logró cumplir con los objetivos establecidos y proporcionar una solución funcional para la edición de artículos en formato LaTeX.

\subsection{Trabajos Futuros}
Aunque el proyecto ha alcanzado sus objetivos principales, aún existen áreas que pueden mejorarse y funcionalidades adicionales que pueden agregarse en el futuro. Algunas de las posibles mejoras y trabajos futuros incluyen:

\begin{itemize}
\item Implementación de autenticación con Google para permitir el inicio de sesión y el registro mediante cuentas de Google, lo que mejoraría la experiencia del usuario y aumentaría la seguridad del sistema.
\item Incorporación de funciones avanzadas de LaTeX, como soporte para ecuaciones matemáticas complejas, referencias cruzadas y bibliografía automática, para proporcionar a los usuarios herramientas más avanzadas para la creación de artículos científicos.
\item Desarrollo de una función de navegación por bloques en el editor de artículos, que permita a los usuarios moverse fácilmente entre diferentes secciones y elementos del documento, facilitando la organización y edición del contenido.
\item Optimización del tiempo de compilación de los documentos LaTeX, mediante la implementación de técnicas de paralelización o la optimización de los procesos de compilación, para reducir el tiempo necesario para generar los archivos PDF.
\end{itemize}

Estas mejoras y trabajos futuros pueden contribuir a mejorar la funcionalidad, usabilidad y rendimiento del sistema, permitiendo así satisfacer mejor las necesidades de los usuarios y proporcionar una experiencia más fluida. El proyecto tiene un gran potencial para seguir creciendo y evolucionando en el futuro, se espera que el desarrollo realizado hasta el momento sea un punto de partida para futuras mejoras y actualizaciones.



