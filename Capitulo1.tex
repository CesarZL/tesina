\section{Introducción}
En el ámbito académico y científico, la redacción de artículos es esencial, pero enfrenta un desafío y es que cada revista requiere un formato específico que debe seguirse al pie de la letra para que el artículo sea aceptado. El objetivo primordial es optimizar el tiempo dedicado a adaptar el contenido a cada formato. Para abordar este desafío, se ha desarrollado una aplicación web que simplifica y agiliza la creación de artículos científicos al permitir a los usuarios seleccionar y utilizar plantillas prediseñadas.

\subsection{Antecedentes del proyecto}
Previo a abordar la definición del problema y la justificación del proyecto, es importante contextualizar la situación que dio origen a la idea de desarrollar la aplicación web. En este sentido, se ha identificado que la redacción de artículos científicos es una tarea común en el ámbito académico y científico. Sin embargo, esta tarea puede resultar tediosa y complicada, ya que cada revista científica tiene un formato específico que debe seguirse al pie de la letra para que el artículo sea aceptado. Por lo tanto, los autores deben dedicar tiempo y esfuerzo a adaptar el contenido a cada formato, lo que resulta en una práctica poco eficiente. Anteriormente se había desarrollado una aplicación similar que funcionaba en un entorno de escritorio, pero no era accesible desde cualquier dispositivo con conexión a Internet. Por lo tanto, se decidió desarrollar una aplicación web que permitiera a los usuarios seleccionar y utilizar plantillas prediseñadas para crear artículos científicos de forma rápida y sencilla.

\subsection{Definición del problema y justificación del proyecto}
Se ha identificado la necesidad de mejorar la eficiencia en la redacción de artículos científicos al simplificar y agilizar el proceso de adaptación del contenido a los formatos específicos de las revistas científicas. Para abordar este desafío, se ha desarrollado una aplicación web que permite a los usuarios seleccionar y utilizar plantillas prediseñadas para crear artículos científicos de forma rápida y sencilla. La aplicación web es accesible desde cualquier dispositivo con conexión a Internet, lo que facilita su uso en cualquier momento y lugar. Además, la aplicación web es fácil de usar y cuenta con una interfaz intuitiva que permite a los usuarios crear artículos científicos sin necesidad de conocimientos técnicos avanzados.

\subsection{Planteamiento del problema}
La redacción de artículos científicos en el ámbito académico y científico es una tarea común, pero puede resultar tediosa y complicada debido a los formatos específicos de las revistas científicas. Los autores deben dedicar tiempo y esfuerzo a adaptar el contenido a cada formato, lo que resulta en una práctica poco eficiente. Por lo tanto, se ha identificado la necesidad de mejorar la eficiencia en la redacción de artículos al simplificar y agilizar el proceso de adaptación del contenido a los formatos específicos de las revistas. 

\subsection{Justificación del proyecto}
La importancia de dedicar tiempo y esfuerzo a la redacción de artículos científicos radica en la necesidad de comunicar los resultados de la investigación de forma clara y precisa. Por lo tanto, se propuso desarrollar una aplicación web que no solo mejorará la agilidad en la gestión de artículos, sino que también permitirá a los autores centrarse en el contenido en lugar de en el formato.

\subsubsection{Actividades realizadas durante la estadía}
Durante la estadía se llevó a cabo un análisis de la problemática principal y se identificaron las necesidades del usuario llevando a cabo la recolección de requisitos y la definición de los objetivos del proyecto, posteriormente se realizáron las actividades de diseño y desarrollo de la aplicación web, pasando por todas las etapas del ciclo de vida del software. Finalmente, se llevó a cabo la implementación de la aplicación web y se realizaron pruebas de usabilidad para garantizar su funcionamiento.

\subsection{Objetivo General}
Desarrollar e implementar una aplicación web que permita a los usuarios seleccionar y utilizar plantillas prediseñadas para crear artículos científicos de forma rápida y sencilla.

\subsection{Objetivos Específicos}
Los siguientes objetivos no tienen un orden de prioridad, ya que todos son igual de importantes para el desarrollo del proyecto.

\begin{itemize}
    \item Establecer los requisitos del sistema basados en una revisión detallada de las necesidades del usuario para el procesamiento de documentos LaTeX.
    \item Investigar y seleccionar las tecnologías más adecuadas para el desarrollo de la aplicación web y el procesamiento de documentos tex, asegurando escalabilidad y eficiencia.
    \item Diseñar e implementar la base de datos para almacenar plantillas, información de coautores y documentos generados, optimizando el acceso y la gestión de datos.
    \item Desarrollar la arquitectura de la aplicación web con Laravel, definiendo claramente la interacción entre componentes y garantizando una integración entre componentes.
    \item Crear una interfaz de usuario intuitiva y fácil de usar, utilizando herramientas modernas como Tailwind CSS y Editor.js, para mejorar la experiencia del usuario.
    \item Configurar el entorno de desarrollo y establecer un flujo de trabajo eficiente que facilite el desarrollo, la colaboración y la gestión del código.
    \item Implementar funcionalidades para la carga de plantillas LaTeX, gestión de coautores y generación automática de documentos basados en archivos y datos proporcionados.
    \item Integrar un sistema de procesamiento de documentos tex utilizando Google Gemini para formatear secciones específicas y generar documentos de alta calidad automáticamente.
    \item Realizar pruebas para garantizar la funcionalidad correcta y la estabilidad del sistema.
    \item Identificar y corregir errores en el código, asegurando la seguridad y la estabilidad de la aplicación.
    \item Ejecutar pruebas de rendimiento para optimizar la respuesta de la aplicación y su capacidad de manejar grandes volúmenes de datos.
    \item Configurar y desplegar la aplicación en un servidor de Digital Ocean.
    \item Realizar pruebas finales en el entorno de producción para asegurar el correcto funcionamiento.
\end{itemize}


\subsection{Alcances y limitaciones del proyecto}
El proyecto se enfocará en el desarrollo e implementación de una aplicación web que permita agilizar la creación de artículos científicos al permitir a los usuarios seleccionar y utilizar plantillas de diferentes revistas científicas. Al ser una aplicación web estará limitada a la infraestructura y recursos disponibles en el servidor de producción. A continuación, se detallan los alcances y limitaciones del proyecto:


\begin{itemize}
    \item Alcances:
    \begin{enumerate}
    \item Desarrollo e implementación de una aplicación web que facilite a los usuarios la selección y utilización de plantillas prediseñadas para crear artículos científicos de manera rápida y sencilla, sin necesidad de conocimientos técnicos avanzados.
    \item La aplicación web será accesible desde cualquier dispositivo con conexión a Internet, permitiendo su uso en cualquier momento y lugar.
    \item Implementación de un sistema de gestión de plantillas y artículos que permita a los usuarios subir, seleccionar y utilizar plantillas de diferentes revistas científicas, así como crear, editar y eliminar artículos científicos.
    \item Integración de un sistema de exportación que posibilite a los usuarios exportar los artículos en formato PDF.
    \item Inclusión de funcionalidades para la gestión y creación de coautores, referencias bibliográficas y figuras.
    \item Implementación de autenticación de usuarios y un sistema de gestión de permisos para garantizar la seguridad y control de acceso.
    \end{enumerate}
    \item Limitaciones:
    \begin{enumerate}
    \item La aplicación web estará limitada por la infraestructura y recursos disponibles en el servidor de producción.
    \item Ausencia de sistemas de revisión por pares, traducción automática y detección de plagio.
    \item Falta de sistemas de recomendación para revistas científicas, coautores y referencias bibliográficas.
    \item No se incluirán funcionalidades avanzadas para la creación de gráficos, tablas y ecuaciones.
    \item La aplicación web no estará disponible en otros idiomas, solo en español.
    \item No se incluirán funcionalidades para la importación de artículos desde otros formatos.
    \item La aplicación web no contará con un sistema de notificaciones para alertar a los usuarios sobre eventos importantes.
    \item No se incluirán funcionalidades para la creación de presentaciones y pósters científicos.
    \end{enumerate}
    \end{itemize}

\subsection{Organización del Documento de Tesina}
El documento de tesina está estructurado de la siguiente manera: en el Capítulo 2 se presenta el marco teórico, donde se abordan los conceptos y tecnologías utilizadas en el desarrollo de la aplicación web. En el Capítulo 3 se describe el sistema propuesto, incluyendo el análisis de requerimientos, la arquitectura del sistema, el diseño del sistema y los algoritmos para el procesamiento de documentos. En el Capítulo 4 se detalla la metodología de desarrollo utilizada, que incluye la planificación del proyecto, la gestión de riesgos y la implementación de la aplicación web. En el Capítulo 5 se presentan los resultados obtenidos, que incluyen pruebas de usabilidad, pruebas de rendimiento y la implementación de la aplicación web en un entorno de producción. Finalmente, en el Capítulo 6 se presentan las conclusiones y sugiere trabajos futuros para mejorar la aplicación web.