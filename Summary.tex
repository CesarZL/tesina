In this work, the objective was to create a web application for the management, creation, and visualization of scientific articles based on templates selected by the user. The web application was developed using the Laravel framework, with the aim of streamlining and automating the process of creating scientific articles, allowing users to focus on the content without worrying about the structure and format of the document.

During the development of the application, various activities were carried out, including the analysis of system requirements, the design of the application architecture, the implementation of the required functionalities, testing, and project documentation. These activities resulted in a functional web application that meets the set objectives and can be used by researchers and academics to create scientific articles quickly and easily.

This document will present in detail all the activities that were carried out to fulfill this project, as well as the results obtained.