En este trabajo se tuvo como objetivo realizar una aplicación web para la gestión, creación y visualización de artículos científico a partir de plantillas seleccionadas por el usuario. La aplicación web fue desarrollada utilizando el framework Laravel, esto con el objetivo de agilizar y automatizar el proceso de creación de artículos científicos, permitiendo a los usuarios enfocarse en el contenido de los mismos sin preocuparse por la estructura y formato del documento. 

Durante el desarrollo de la aplicación se realizaron diversas actividades, incluido el análisis de los requisitos del sistema, el diseño de la arquitectura de la aplicación, la implementación de las funcionalidades requeridas, la realización de pruebas y la documentación del proyecto. Estas actividades resultaron en una aplicación web funcional que cumple con los objetivos planteados y que puede ser utilizada por investigadores y académicos para la creación de artículos científicos de forma rápida y sencilla.

En este documento se presentarán con detalle todas las actividades que fueron desarrolladas para el cumplimiento de este proyecto, así como los resultados obtenidos.

